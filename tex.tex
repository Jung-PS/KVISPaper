\documentclass[12pt,twocolumn,a4paper]{article}
\usepackage[english]{babel}
\usepackage[utf8x]{inputenc}
\usepackage[T1]{fontenc}
\usepackage{float}
\usepackage[a4paper,top=3cm,bottom=2cm,left=3cm,right=3cm,marginparwidth=1.75cm]{geometry}
\usepackage{amsfonts}
\usepackage{amsmath}
\usepackage{graphicx}
\usepackage[colorinlistoftodos]{todonotes}
\usepackage[colorlinks=true, allcolors=blue]{hyperref}
\setlength{\marginparwidth}{2cm}
\usepackage{mathtools}
\usepackage{apalike}
\usepackage{babel} 
\usepackage{comment}
\title{
%\vspace{-1in} 	
\usefont{OT1}{bch}{b}{n}
\normalfont \normalsize \textsc{Kamnoetvidya Science Academy Thematic paper} \\ [14pt]
\LARGE The Effects of Political Spectrum on Economic Outcomes\\
}


\usepackage{authblk}
\usepackage{xurl}
\author[1]{Angkul Pitaksa}
\author[1]{Khunanon Itthinanthawan}
\author[1]{Natthakan Losatiankit}
\author[1]{Paponpat Siriruamsap}
\author[1]{Phingphu Wesaratprasert}
\author[1]{Sathawit Sirichantasing}

\affil[1]{\small{SO30102, Kamnoetvidya Science Academy}}
\begin{document}
\maketitle

\selectlanguage{english}
\begin{abstract}
A political spectrum is a characterization of political positions, usually laid down in coordinate system. Various distinctive types of characterization are available.

We focus solely on communism-capitalism economic scale, as it can be easily visualized due to the existence in modern society.

In this thematic paper, we compare and contrast different economic system with the concept, benefits, drawback, and the four main aspects: economic growth, economic stability, political stability, and equality.
\end{abstract} 

\section*{Keywords}


\section{Introduction}
An economic system is a model of an economy, including roles of government, decision-making activities and regulation of production, resources allocation, and distribution.

Various economic system can be visualized as a one-dimensional scale ranging from the left, pure communism to the right, pure free market capitalism. 

\begin{comment}
The terms left and right originated from French revolution, in which, the supporters of \textit{ancien régime} sits to the right side of the National Assembly and the supporters of revolution sits on the left side. Thus, right side can be considered as conservative, while left side can be viewed as progressive.
\end{comment}

\section{Overview of Capitalism and Communism}
In the left-right economic system, capitalism is deemed as the antipode of communism due to the contrasting of ideologies.

\subsection{Pure Capitalism}
Capitalism focused on the concept of private ownership, the capital as the means of production. Competitiveness of market is extreme and the government intervention is minimal.
\subsection{Pure Communism}
Communism centered on shared ownership as the means of production, distribution and exchange of products is based on need. Ideally, Society is classless, the concept of private property and money is eliminated.\newline


Situated between capitalism and communism, are liberalism, centralism and socialism respectively.

\section{Pure Capitalism}
Pure capitalism is a theoratical economic system, in which, the market is extremely free and competitive, government intervention is non-existence and the market is fully regulated with market mechanism.

This system is deeply rooted by \textit{Laissez-faire}, transliterated to \textit{let do}, in this system, individual is believed to have a natural right to freedom, therefore, mareket should naturally be competitive. Due to this essence, prices of goods and services are fully determined by supply and demand, the market mechanism. Furthermore, government intervention should not be allowed, as capitalism believes that intervention can only led to the decreased efficiency of self-regulated market mechanism.

Capitalism is profit-motived, private companies goal is to maximize profit. Thus, the investment of financial asset fueled economic's prosperity.

The only obligation of government are law-enforcement activities, national security to assist private companies.

\subsection{Benefits}
Pure capitalism benefits include rapid technological advancement due to competitive nature, products and services are constantly adapted to suit the people. Creativity can be freely expressed without restraint, caused by the bureaucratic system.

\subsection{Drawback}
In pure capitalism, due to the lack of government intervention, extreme wealth inequality, bourgeoisie system, and worker abuse can occur due to profit prioritization. Furthermore, monopoly can transpise due to amalgamation of businesses and acquisition of smaller companies. If monopoly happens, because of the lack of competition and substitute good, this can lead to high monopoly profit margin.

\subsection{The Four Aspects}
\subsubsection{Economic Growth}
The economic growth of pure free market, in long term, is normally substantial as the private companies have freedom and ability to take actions and create new innovation with little or no restriction. Aside from that, the competitive nature of the market also have a great emphasis towards the economic growth.

In short term, due to the lack of safety net and government intervention, business cycle can happens repeatedly, recession and recovery follow.
\subsubsection{Economic Stability}
Due to the lack of state intervention and corrective measures, business cycle will repeatedly occur, expansion and recession, crisises can followed such as the great recession caused by failures in financial regulation of U.S. Federal Reserve and housing bubble.

\subsubsection{Political Stability}
As government dependencies is reduced, politically, it is quite stable. there are less corruptions and more democratic support.

\subsubsection{Equality}
Equality level in pure free market system can be low, due to discrimination of workers, profit prioritization of companies, and lack of social welfare system. However, in some cases, this model can gives everyone freedom and opportunity to compete.

\subsection{Examples}
Although, no country operates a truly pure free market, as government intervention take place in some extent, countries which are considered to have high level of freedom and minimal regulations are Singapore, Hong Kong, and the United States.
\begin{comment}
\begin{figure}[H]
\centering
\includegraphics[width=\linewidth]{images/hash.png}
\caption{A Hash Table}
\end{figure}
\end{comment}
\section{Conclusion}





%% Please read the following comments for instructions on formatting your references.
%% The provided style file formats references in the required Vancouver style
%% This is a useful guide to formatting bibliographies using LaTeX (https://en.wikibooks.org/wiki/LaTeX/Bibliography_Management)

%% References with BibTeX database:
%%Bib\TeX{} is a bibliographic tool that is used with \LaTeX{} to help organize the user's references and create a bibliography. This short tutorial explains how to use BibTex files [https://www.latex-tutorial.com/tutorials/beginners/latex-bibtex/].

%%For references without a BibTeX database:
%%Your line of code will start with the \bibitem{cite_key} command. This is a xunique identifier for a reference, such as #\bibitem{Lindquist95} or  #\bibitem{Smithetal2002}. Following the #\bibitem{cite_key} command is the actual reference. Here you should include all relevant details for the reference e.g. authors, title, year of publication, etc. The \bibliographystyle{vancouver} will format these references to the correct Vancouver style - as required by the STEM Fellowship Journal.

%% \bibitem must have the following form:
%%   \bibitem{key}...
%%

% \bibitem{}
% Below is an example.

%%\bibitem{Smithetal2002}
\bibliographystyle{apalike}
\bibliography{references}
\nocite{*}
\end{document}